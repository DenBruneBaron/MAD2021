\section{Problem 4}
\subsection{(a)}
In this exercise I'm asked to choose the null hypothesis.
\\
My Null hypothesis is:, $ H_0 : \mu_0 = 0 $
My alternative hypothesis is, $H_A : \mu \neq \mu_0$
\\
Which means, that I assume that there is no difference in flowering time, since the value $X_3 - Y_3 = -0.5$ shows that
the scientists claim does not hold for all of the samples. With the specified
alternative hypothesis, I would have to perform af two-sided t-test
\\
$ \mu_0 = 0 $ since the assumption is, that there is no difference between the two types of flowers

\subsection{(b)}
Performing the corrosponding t-test (Assuming that I have to perform the corrosponding test using my claim from (a)).
\\
Following the "six steps" from the lecture. I will be starting from step 3 since both step one and two are defined in question (a).
\\
The dataset is $ X_i - Y_i $, which gives me

\begin{center}
    \begin{tabular}{ |c|c|c|c|c| } 
     \hline
      1 & 0.5 & -0.5 & 1.5 & 0.5\\ 
     \hline
    \end{tabular}
\end{center}

\noindent Calculating the observed mean:\\
$ \frac{1 + 0.5 - 0.5 + 1.5 + 0.5}{5} = 0.6 $
\\
\\
Calculating the standard deviation for my sample \\
$ S = \sqrt{Var(Z)}$ \\
\\
$ Var(Z) = \sum_{i=1}^{5} \frac{(x_i - \overline{x})^2}{n-1} $ \\
\\
$ Var(Z) = \frac{(1 - 0.6)^2 + (0.5 - 0.6)^2 + (-0.5 - 0.6)^2 + (1.5 - 0.6)^2 + (0.5 - 0.6)^2}{4} = \frac{2.2}{4} = 0.55 $  
\\
\\
$ S = \sqrt{0.55} = 0.7416$
\\
\\
$t = \frac{\overline{x} - \mu_0}{S \sqrt{n}} = \frac{0.6 - 0}{0.7416 \sqrt{5}} = \frac{0.6}{0.3317} \approx 1.81 $
\\
\\
I've calculated $c_{1}$ and $c_{2}$ using the following code I made.

\begin{minted}[linenos, bgcolor = bg]{python}
    from scipy.stats import t

    # Mean of my sample
    mean = 0.6

    # Samples (5 flowers)
    n_samples = 5

    # I need to divide alpha since I'm making use of
    # the two-side test
    alpha_val = 0.05 / 2
    std_deviation = 0.7416

    # Performing the t.ppf, in order to find c1 and c2 
    c1 = t.ppf(alpha_val, n_samples-1,
               loc= mean, scale = std_deviation)

    c2 = t.ppf((1 - alpha_val), n_samples-1, 
               loc= mean, scale = std_deviation)

    print("c1: lower_cutoff", c1)
    print("c2: upper_cutoff", c2)
\end{minted}

\noindent which gives me:
$$c_1 \approx -1.4590 \quad \quad \text{and} \quad \quad c_2 \approx = 2.6590 $$
\\
Since $c_1 < t < c_2 $ meaning that t is in the "acceptance" region, I can accept that 
the flowering time with the two types of flowers appears to be indifferent.

\subsection{(c)}
No, it is not that simple. If the scientist were to multiply the whole dataset with
a number $ k $ it would change the calculations of $ t $.
The standard deviation will cause problems since the degrees of freedom will not be the same

$$ Var^* = \sum_{i = 1}^{k \cdot n} \frac{(x_i - \mu)^2}{kn-1} \neq k \cdot \sum_{i = 1}^{ n} \frac{(x_i - \mu)^2}{n-1}$$
