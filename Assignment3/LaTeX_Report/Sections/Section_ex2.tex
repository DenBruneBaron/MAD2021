\section{Problem 2}
assesses the given claim $E[(X-\mu)^{4}] \geq \sigma^{4}$
\\
\\
X has the mean $\mu$ and the variance $\sigma^{4}$ which can be rewritten as $\sigma^{4} = (Var(X))^2$
$E[(X-\mu)^{4}]$ kan be rewritten as $E(g(x))$ where $g(x) = (x - \mu)^{4}$
\\
It is possible to Jensen's inequality if $g''(x) \geq 0$\\
$g(x) = (x-\mu)^{4}$\\
$g(x) = 4(x-\mu)^{3}$\\
$g(x) = 12(x-\mu)^{2}$\\
g(x) is convex, since the second derivative of the function is quadratic. Hence it will always be
greater than zero. Which means that it its possible to make use of Jensen's inequality.
\\
% By definition the expected value of a linear function is equal to the same linear function applied to the 
% expected value  
$E(cX - \mu)^4 \geq (E(X - \mu))^4$\\
$(E(X - \mu))^4$\\
$((E(X - \mu))^2)^2 = (Var(X))^2$\\
\\
thus the claim is true, and shown by Jensen's inequality
$$E[(X-\mu)^{4}] \geq \sigma^{4}$$