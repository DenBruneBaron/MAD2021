\section{Problem 2}
\section{Problem 2}
assesses the given claim $E[(X-\mu)^{4}] \geq \sigma^{4}$
\\
\\
X has the mean $\mu$ and the variance $\sigma^{4}$ which can be rewritten as $\sigma^{4} = (Var(X))^2$
$E[(X-\mu)^{4}]$ kan be rewritten as $E(g(x))$ where $g(x) = (X-\mu)^{4}$
\\
It is possible to Jensen's inequality if $g''(x) \geq 0$
$g(x) = (X-\mu)^{4}$
$g(x) = 4(X-\mu)^{3}$
$g(x) = 12(X-\mu)^{2}$
g(x) is convex, since the second derivative of the function is quadratic. Hence it will always be
greater than zero.
 