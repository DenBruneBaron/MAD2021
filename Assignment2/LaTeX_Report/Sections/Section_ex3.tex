\section{Problem 3}
\subsection{(a)}
The a PDF is the derivative of a CDF.
\\
The function $e^{- \beta x^{\alpha}}$ is a composition of f(g(x)). 
Thus the after applying the chainrule and deriving where $g(x) = - \beta x^{\alpha} $ :
$$\frac{d}{dx} f(g(x)) = \frac{d}{dx} (- \beta x^{\alpha} \cdot e^{{g(x)}}) = - \beta \alpha x^{\alpha-1} \cdot e^{- \beta x^{\alpha}} $$
\\
Thus the PDF is:
\[f(x) \begin{cases}
    - \beta \alpha x^{\alpha-1} \cdot e^{- \beta x^{\alpha}} & \quad \text{if} \quad x > 0 \\
    0 & \quad \text{otherwise} \\
 \end{cases}
\]

\subsection{(b)}
In this question there are two subquestions.
\begin{enumerate}
    \item What is the probability that the chip works longer than four years?
    \item What is the probability that the chip stops working in the time interval [5;
    10] years?
\end{enumerate}
I'm given two values for $\alpha$ and $\beta$ stubstituting these values into the original
function I have:
$$ f(x) = 1 - e^{-\frac{1}{4}x^2} = 1 - e^{-\frac{x^2}{4}} $$
\\
To answer the first question I can simply calculate f(5). Since I'm asked to answer the probability
for the chip to live \emph{more} than four years. It is important to note that when the value = 1 the
chip is dead. Otherwise, the chip would become better over time.
$f(5) = 0.998069 $ This means, that after 5 years, the chip is almost certanly dead.
\\
\\
To calculate the probability that the chip will die somewhere between the interval $[5;10]$
I can do it like so:
$$F(10)-F(5) = 0.0183156$$ This means that the chip has $\approx 1.83\%$ chance of surviving. So to make this 
more readble I can say:
\\
$$1 - (F(10)-F(5)) \approx 98.16\% \quad \text{chance of death}$$ in the interval from 5 to 10 years.
\subsection{(c)}
Finding the median of the function. I can simply set it equal to $\frac{1}{2}$ and solve it for x.
$$1 - e^{- \beta x^{\alpha}} = \frac{1}{2}$$
simple rewriting
$$1 - \frac{1}{2} = e^{- \beta x^{\alpha}}$$
multiplying with $ln()$ to remove the exponent
$$ ln(\frac{1}{2}) = - \beta x^{\alpha} $$
dividing with $-\beta$
$$\frac{ln(\frac{1}{2})}{-\beta} = x^{\alpha} $$
$ln(\frac{1}{2})$ is the same as $-ln(2)$ so rewriting and removing the minuses
$$\frac{ln(2)}{\beta} = x^{\alpha} $$
taking $ln(x^{\alpha})$ to move the $\alpha$
$$\ln( \frac{ln(2)}{\beta} ) = \alpha \cdot ln(x) $$
moving alpha
$$\frac{\ln( \frac{ln(2)}{\beta} )}{\alpha} = ln(x)$$
We already know that in order to remove the exponenet, we can use ln. so to remove ln
I can add back the exponent, which gives me the answer
\\
\textbf{Answer:}
$$e^{\frac{\ln( \frac{ln(2)}{\beta} )}{\alpha}} = x$$