\section{Problem 3 - Principal Component Analysis}
\subsection{1) - Mean point of the data}
Step one is to calculate the mean point of the data with respect to both x and y

$$\bar{x} = \frac{1}{8} (0.6 + 0.5 + 1.1 + (-0.5) + 0.8 + 0.2 + (-0.1) + 1.0) = 0.4500$$

$$\bar{y} = \frac{1}{8} (1.1 + 1.0 + 2.0 + 0.2 + (-0.1) + (-0.1) + (-1.5) + 2.5) = 0.6375$$

Step two is to normalize the dataset. This is done by taking each original point
and subtract the mean.
$p - [\bar{x}, \bar{y}]$

Normalizing each point
\begin{align}
    &p1 = (0.6 , 1.1) - [0.4500, 0.6375] \quad = \quad [0.1500, 0.4625] \\
    &p2 = (0.5 , 1.0) - [0.4500, 0.6375] \quad = \quad [0.0500, 0.3625] \\
    &p3 = (1.1 , 2.0) - [0.4500, 0.6375] \quad = \quad [0.6500, 1.3625] \\
    &p4 = (-0.5 , 0.2) - [0.4500, 0.6375] \quad = \quad [-0.9500, -0.4375] \\
    &p5 = (0.8, -0.1) - [0.4500, 0.6375] \quad = \quad [0.3500, -0.7375] \\
    &p6 = (0.2 , -0.1) - [0.4500, 0.6375] \quad = \quad [-0.2500, -0.7375] \\
    &p7 = (-0.1 , -1.5) - [0.4500, 0.6375] \quad = \quad [-0.5500, -2.1375] \\
    &p8 = (1.0, 2.5) - [0.4500, 0.6375] \quad = \quad [0.5500, 1.8625] 
\end{align}

Step three is to calculate the covariances for XX, XY and YY using the normalized values for p1-p8


\begin{equation}
    \begin{split}
        cov(xx) = & \frac{1}{8} \Big(0.15^2 + 0.05^2 + 0.65^2 + (-0.95)^2 \\
                  & + (-0.25)^2 + (-0.55)^2 + 0.55^2 \Big) \approx 0.27
    \end{split}
\end{equation}

\begin{equation}
    \begin{split}
        cov(yy) = & \frac{1}{8} \Big(0.4625^2 + 0.3625^2 + 1.3625^2 + (-0.4375)^2 + \\
                  &(-0.7375)^2 + (-0.7375)^2 + (-2.1375)^2 + 1.8625^2 \Big) \approx 1.44
    \end{split}
\end{equation}

\begin{equation}
    \begin{split}
        cov(xy) = & \frac{1}{8} \Big( (0.15\cdot 0.4625 + 0.05 \cdot 0.3625 + 0.65 \cdot  1.3625 + ((-0.95)\cdot (-0.4375))\\
                  &  + 0.35 \cdot (-0.7375) + ((-0.25)\cdot (-0.7375)) + ((-0.55)\cdot (-2.1375))\\
                  &  + 0.55 \cdot 1.8625) \Big) \approx 0.44
    \end{split}
\end{equation}

having done the three calucaltions I can now form the covariance matrix inserting the calculated
values into their respectable place:
$$
cov = \begin{pmatrix}
    xx & xy \\
    xy & yy
 \end{pmatrix}
 = 
$$

$$ cov = \begin{pmatrix}
            0.27 & 0.44 \\
            0.44 & 1.44
         \end{pmatrix}$$


\subsection{2) - Two eigenvalues}
I order to find the eigenvalues and the eigenvectors i need to solve \textbf{\textit{Characteristic equation}} of the
covariance matrix.
The Characteristic equation is denoted as: 
$$det(A - \lambda I) = 0$$

To calculate the determinant you calculate the following:
$$
determinant = \begin{pmatrix}
                a & b \\
                c & c
              \end{pmatrix}
= ad - bc
$$
Solving the system equation that I get from uising the Characteristics equation on the covariance matrix
i get the following eigenvalues.
$$\begin{pmatrix}
    0.27 - \lambda & 0.44 \\
    0.44 & 1.44 - \lambda
\end{pmatrix}
= 0
$$
First I get the equation by calulating the determinant
$$det(cov) \rightarrow ((0.27 - \lambda) \cdot (1.44 - \lambda)) - (0.44 \cdot 0.44) $$
Then setting the equation equal to zero and solving it will give me the two values for $\lambda$ which is the
eigenvalues.
$$((0.2675 - \lambda) \cdot (1.44 - \lambda)) - (0.44 \cdot 0.44) = 0$$
$$eigenvalue_1 = 1.59 \quad \quad eigenvalue_2 = 0.12$$


\subsection{3) - Two eigenvectors}
I order to find the eigenvectors I use the definition of an eigenvector $A\bar{\textbf{x}} = \lambda\bar{\textbf{x}}$
and solve this this equation.
$$cov \cdot \bar{x} = \lambda \cdot \bar{x}$$

finding the eigenvectors, using the equation stated above, I have two equations systems.
\begin{align*}
    & 0.27 x_1 + 0.44 x_2 = 1.59 x_1 \\
    & 0.44 x_1 + 1.44 x_1 = 1.59 x_2 \\
\end{align*}
and 
\begin{align*}
    & 0.27 x_1 + 0.44 x_2 = 0.12 x_1 \\
    & 0.44 x_1 + 1.44 x_1 = 0.12 x_2 \\
\end{align*}

solving these systems of equations using basic equation arithmetic

\begin{align*}
    & 0.44 x_2 = (1.59 - 0.27) x_1 \\
    & x_2 = \frac{1.32}{0.44} x_1 \\
    & x_2 = 3 x_1 \\
    \\
    & \text{thus, if} \quad x_1 = 1 \\
    & \text{then} \quad \quad x_2 = 3 
\end{align*}
Now that I have the values for $x_1 and x_2$
I can find the actual eigenvector, by finding the unit vector for the vector     
$
\bar{x} = \begin{bmatrix}
    1 \\
    3 \\
  \end{bmatrix}
$
to find the unit vector I'm using the formula $\frac{1}{||\bar{v}||} \cdot \bar{v}$
which is equivalent to
\begin{align*}
    &\frac{1}{\sqrt{(1^2 + 3^2)}} \cdot 
                            \begin{bmatrix}
                                1 \\
                                3 \\
                            \end{bmatrix} \\
    \\
                            &\frac{3}{\sqrt{(1^2 + 3^2)}} \cdot 
                            \begin{bmatrix}
                                1 \\
                                3 \\
                            \end{bmatrix} \\
    \\
    & x_1 = 0.316 \quad \quad x_2 = 0.948
\end{align*}
I've shown the calulations for the first system, with the first equation and I think
it's fair to say that finding the second eigenvector is by doing excatly the same, with
the only exception being that lambda now is the second eigenvalue.
Thus my eigenvectors are:
\begin{align*}
    Eigenvector_1 = \begin{bmatrix}
        0.316 \\
        0.949 \\
    \end{bmatrix} 
\quad 
    Eigenvector_2 = \begin{bmatrix}
        -0.949 \\
        0.316 \\
    \end{bmatrix} \\
\end{align*}

\noindent note that signs of the eigenvector can be different. Depending on how you choose to
solve your equation systems.

\newpage