\section{Problem 2}
In order to solve these exercises I will make use of \emph{Tabel 1.4} in the book "A first couse in machine learning"
\subsection{(a)}
$f(\overline{x})=\overline{x}^{T}\overline{x}+c$
\\
\textbf{Answer:}
\\
Here we have the "third case" of the table, 
$\overline{x}^{T}\overline{x} \Rightarrow \nabla f(\overline{x}) = 2\overline{x}$\\
The constant c is simply discarded, when differentiating.
\subsection{(b)}
$f(\overline{x})=\overline{x}^{T}\overline{b}$
\\
\textbf{Answer:}
\\
Here we have the "first case" of the table, 
$\overline{x}^{T}\overline{b} \Rightarrow \nabla f(\overline{x}) = \overline{b}$


\subsection{(c)}
$f(\overline{x})=\overline{x}^{T}A\overline{x}+\overline{b}^{T}\overline{x}+c$
\\
\textbf{Answer:}
\\
Here we have the a combination of "fourth case" and the "second case" from the table.
\\
$\overline{x}^{T}A\overline{x} \Rightarrow 2A\overline{x} $\\
and\\
$\overline{b}^{T}\overline{x} \Rightarrow \overline{b} $\\
Combining the two, we get the result: $\nabla f(\overline{x}) = 2A\overline{x} + \overline{b}$\\
The constant c is simply discarded, when differentiating.

